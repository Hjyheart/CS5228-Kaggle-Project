\documentclass[conference]{IEEEtran}
\IEEEoverridecommandlockouts
% The preceding line is only needed to identify funding in the first footnote. If that is unneeded, please comment it out.
\usepackage{cite}
\usepackage{amsmath,amssymb,amsfonts}
\usepackage{algorithmic}
\usepackage{graphicx}
\usepackage{textcomp}
\usepackage{xcolor}
\usepackage{booktabs}
\usepackage{tabularx}  
\def\BibTeX{{\rm B\kern-.05em{\sc i\kern-.025em b}\kern-.08em
    T\kern-.1667em\lower.7ex\hbox{E}\kern-.125emX}}

\begin{document}

\title{CS5228 Final Project Report\\
GROUP NAME}

\author{\IEEEauthorblockN{CHAN CHEAH CHA}
\IEEEauthorblockA{
A0189006A \\
E0324590}
\and
\IEEEauthorblockN{HONG JIAYONG}
\IEEEauthorblockA{
 \\
}
\and
\IEEEauthorblockN{PHANG DAO YI JOSEPH}
\IEEEauthorblockA{
 \\
}
}

\maketitle

\begin{abstract}
    
\end{abstract}
% Section 1: Introduction
\section{Introduction}
\label{sec:introduction}
Introduction.
\subsection{Motivation}
Motivate and outline the goals and questions you address. Why is this work important, what are the challenges, will benefits from the results. In short, what problem are you trying to solve? For example, simply aiming for a top rank on Kaggle is not a sufficient motivation :).

% Section 2: Data Preparation
\section{Data Preparation}
\label{sec:data_preparation}
In this section, we explain the data preparation process, including checking data quality, exploring the data, and preprocessing it for modeling.

\subsection{Understanding the Data}
The dataset contains both numerical and categorical attributes, which represent different aspects of the cars listed for resale. 
The training dataset consists of 25,000 records with 30 attributes, while the test dataset consists of 10,000 records with 29 attributes (\texttt{price} target variable excluded).

To better understand the data and its context, we conducted background research focusing on key attributes that influence resale car prices. This analysis will aid in the interpretation of our dataset and guide our preprocessing steps.

\subsubsection{Price-Related Attributes}
The price of a car in the resale market is influenced by several key attributes and and understanding them is essential for accurate price prediction. 
\begin{itemize}
    \item \textbf{Deregistration Value} The car's value at the end of its COE tenure. 
    For PARF cars (less than 10 years old), the deregistration value includes both the PARF rebate and COE rebate. 
    This suggests that PARF cars have higher overall deregistration value compared to COE cars, which may positively impact the resale price. 
    \item \textbf{Depreciation} Depreciation value is directly computed from the resale price and deregistration value over the remaining years of COE left.
\end{itemize}
\subsubsection{Market-Related Attributes}
Market dynamics also play a significant role in determining resale prices. Understanding these attributes is essential for a comprehensive analysis of the resale market.

\begin{itemize}
    \item \textbf{Demand and Supply} The oversupply of similar models can lead to lower prices, particularly for cars commonly used in ride-sharing services. Attributes such as model, make, and manufactured (year) are related to this aspect.
    \item \textbf{Popularity or Rarity of Car Model} Rare or exotic cars may fetch higher prices, making this a potential feature for model classification.
    \item \textbf{Consumer Preferences} Shifting trends towards eco-friendly vehicles can increase demand for hybrid or electric cars, which is linked to the fuel\_type attribute.
    \item \textbf{Regulatory Changes} Fluctuations in COE prices affect different vehicle classes, necessitating potential feature construction from engine\_cap, power, and registration dates.
\end{itemize}

These insights from market-related attributes will enhance our understanding of the pricing dynamics in the resale car market.

{\small Describe the dataset in detail, including the types of attributes (categorical, numerical, etc.), and provide an overview of how these attributes might influence car prices.}

\subsection{Data Quality Check}
\label{subsec:data_quality}
In this dataset, significant missing values were identified across various features, as summarized below:

\begin{itemize}
    \item \textbf{Indicative Price} This feature is completely missing for all records, which is a significant concern as it represents a key aspect of resale price analysis.
    
    \item \textbf{OPC Scheme} and \textbf{Original Registration Date} Both of these features have missing data for nearly all records, with over 98\% missing. This could severely limit our ability to analyze vehicle history and ownership.
    
    \item \textbf{Lifespan} Approximately 90\% of the records lack this information, making it challenging to determine the expected value depreciation over time.
    
    \item \textbf{Fuel Type} Missing for about 76\% of the records, this attribute is essential for understanding consumer preferences and market trends towards eco-friendly vehicles.
    
    \item \textbf{Mileage} With around 21\% of the records missing mileage data, this could impact the resale price estimation as mileage is a significant determinant of car value.
    
    \item \textbf{Other Attributes}: Several other fields exhibit varying degrees of missing data, ranging from 0.03\% to 15.25\%. 
\end{itemize}

{\small Evaluate the quality of the data by checking for missing values, inconsistencies, duplicates, and outliers. Discuss how you addressed any issues related to data completeness and correctness.}

\subsection{Exploratory Data Analysis (EDA)}
\label{subsec:eda}
Perform an initial exploratory analysis to understand the distribution of data and relationships between key variables. This may include descriptive statistics, visualizations (e.g., histograms, scatter plots), and identifying trends or patterns.

\subsection{Data Preprocessing}
\label{subsec:data_preprocessing}
In this section, you will explain the key preprocessing tasks that were applied to the dataset:

\subsubsection{Data Cleaning}
Handle missing values, outliers, and any inconsistencies in the dataset. Discuss the strategies used, such as imputing missing values or removing invalid records.

\subsubsection{Data Reduction}
Reduce the dimensionality of the data by removing redundant or irrelevant features. This could include dropping highly correlated variables or irrelevant fields, as well as feature selection techniques.

\subsubsection{Data Transformation}
Transform the data as needed for modeling. This may involve scaling numerical features, normalizing the data, or applying logarithmic transformations to certain attributes.

\subsubsection{Data Discretization}
Discuss how you discretized continuous features (if necessary), grouping them into categories or bins for better interpretability or model performance.

\subsubsection{One-Hot Encoding (OHE)}
Explain how you converted categorical variables into numerical format using One-Hot Encoding (OHE) to ensure the machine learning models can interpret these variables. Provide details on which categorical variables were transformed and how.

% Section 3: Data Mining Methods
\section{Data Mining Methods}
\label{sec:data_mining_methods}

\subsection{Regression Techniques Applied}
Detail the regression techniques you selected, such as Linear Regression, Random Forest, or XGBoost. Briefly mention how you chose these models.

\subsection{Hyperparameter Tuning}
Describe the process of hyperparameter tuning, including the methods (e.g., grid search, random search) used to optimize the models.

\subsection{Feature Importance}
Explain how you measured the importance of different features in your regression models. This could include methods like permutation importance or looking at feature coefficients in linear models.

\subsection{Model Evaluation}
\label{subsec:model_evaluation}
Evaluate the models using performance metrics such as Mean Absolute Error (MAE), Root Mean Squared Error (RMSE), and R-squared. Compare the results between different models to determine which performed best.

\subsection{Error Analysis}
\label{subsec:error_analysis}
Discuss the error distribution, identifying where the models underperform (e.g., particular car types, extreme mileage values). Analyze the causes of errors and any data or model limitations.

\subsection{Limitations \& Future Work}
\label{subsec:limitations}
Explain the principal limitations of your approach. This might include data constraints, model assumptions, or computational challenges. Suggest potential extensions or improvements for future work, such as more sophisticated feature engineering or applying additional machine learning techniques.

\section{Conclusion}
\label{sec:conclusion}
Summarize the findings of your project, including key insights on predicting car resale prices, feature importance, and the performance of various models.


\newpage
\section{Instructions (TO BE REMOVED)}
The final report will be a PDF document in the format of a scientific paper of at most 8 pages including tables, plots and figures, but excluding references and the appendix. The appendix may contain supplementary content but should be used sparingly. As a rule of thumb, the report should be readable and completely comprehensible without the appendix. The appendix typically may include plots or tables that elaborate on the results of your EDA or your evaluation. For the layout and presentation of the report, we provide templates Download templatesfor Word and LaTeX.
\\
Your report should include the name and student IDs of all team members as well as your team name – pick something cool :). Please also include a breakdown of your workload, i.e., some overview what team member was (mainly) responsible for each part of the project. This can be a table, Gantt chart, etc. to be added to the appendix. While the overall structure of the report is up to you, it should cover the following aspects – although this might differ from your exact project task:
\subsection{Goal}
The goal of this task is to predict the resale price of a car based on its properties 
(e.g., make, model, mileage, age, power, etc). It is therefore first and foremost a regression task. 
These different types of information allow you to come up with features for training a regressor. 
It is part of the project for you to justify, derive and evaluate different features. 
Besides predicting the outcome in terms of a dollar value, other useful results include the importance of 
different attributes, the evaluation and comparison of different regression techniques, an error analysis 
and discussion about limitations and potential extensions, etc.
\subsection{Evaluation}
The evaluation metric for this competition is Root Mean Squared Error (RSME). The RSME is a common metric to evaluate regression tasks. We use the RSME (instead of the Mean Squared Error) so that the error values have the correct unit, which is SGD for this task.
\subsection{Submission Format}
Submission files should contain two columns: Id and Predicted, separated by a comma -- see example-submission.csv for an example. The order of the predictions have to match the order of the test data (test.csv). For example the line with Id=0 should contain the prediction for the first test sample, and so on.
\subsection{Report Structure}
Motivation. Motivate and outline the goals and questions you address. Why is this work important, what are the challenges, will benefits from the results. In short, what problem are you trying to solve? For example, simply aiming for a top rank on Kaggle is not a sufficient motivation :).
\\\\
Exploratory Data Analysis \& Preprocessing. Explain and justify your approach to understand the data, and how it informed your data preprocessing steps (e.g., data reduction, data transformation, outlier removal, feature generation).
\\\\
Data Mining Methods. Describe how you chose and applied appropriate data mining techniques. This description should include which techniques you used, how you chose their hyperparameters, etc. Note that you do not need to explain the techniques themselves. However, in case of more advanced methods or models, you should add relevant references.
\\\\
Evaluation \& Interpretation. Evaluate and compare the performance of different methods. Discuss which method(s) performed best and why. Understand in what cases your methods perform bad, and discuss principle limitations and potential future steps for improvement.


\newpage
\section{ORIGINAL TEMPLATE FOR REFERENCE!!!}
\section{Introduction}
This document is a model and instructions for \LaTeX.
Please observe the conference page limits. 

\section{Ease of Use}

\subsection{Maintaining the Integrity of the Specifications}

The IEEEtran class file is used to format your paper and style the text. All margins, 
column widths, line spaces, and text fonts are prescribed; please do not 
alter them. You may note peculiarities. For example, the head margin
measures proportionately more than is customary. This measurement 
and others are deliberate, using specifications that anticipate your paper 
as one part of the entire proceedings, and not as an independent document. 
Please do not revise any of the current designations.

\section{Prepare Your Paper Before Styling}
Before you begin to format your paper, first write and save the content as a 
separate text file. Complete all content and organizational editing before 
formatting. Please note sections \ref{AA}--\ref{SCM} below for more information on 
proofreading, spelling and grammar.

Keep your text and graphic files separate until after the text has been 
formatted and styled. Do not number text heads---{\LaTeX} will do that 
for you.

\subsection{Abbreviations and Acronyms}\label{AA}
Define abbreviations and acronyms the first time they are used in the text, 
even after they have been defined in the abstract. Abbreviations such as 
IEEE, SI, MKS, CGS, ac, dc, and rms do not have to be defined. Do not use 
abbreviations in the title or heads unless they are unavoidable.

\subsection{Units}
\begin{itemize}
\item Use either SI (MKS) or CGS as primary units. (SI units are encouraged.) English units may be used as secondary units (in parentheses). An exception would be the use of English units as identifiers in trade, such as ``3.5-inch disk drive''.
\item Avoid combining SI and CGS units, such as current in amperes and magnetic field in oersteds. This often leads to confusion because equations do not balance dimensionally. If you must use mixed units, clearly state the units for each quantity that you use in an equation.
\item Do not mix complete spellings and abbreviations of units: ``Wb/m\textsuperscript{2}'' or ``webers per square meter'', not ``webers/m\textsuperscript{2}''. Spell out units when they appear in text: ``. . . a few henries'', not ``. . . a few H''.
\item Use a zero before decimal points: ``0.25'', not ``.25''. Use ``cm\textsuperscript{3}'', not ``cc''.)
\end{itemize}

\subsection{Equations}
Number equations consecutively. To make your 
equations more compact, you may use the solidus (~/~), the exp function, or 
appropriate exponents. Italicize Roman symbols for quantities and variables, 
but not Greek symbols. Use a long dash rather than a hyphen for a minus 
sign. Punctuate equations with commas or periods when they are part of a 
sentence, as in:
\begin{equation}
a+b=\gamma\label{eq}
\end{equation}

Be sure that the 
symbols in your equation have been defined before or immediately following 
the equation. Use ``\eqref{eq}'', not ``Eq.~\eqref{eq}'' or ``equation \eqref{eq}'', except at 
the beginning of a sentence: ``Equation \eqref{eq} is . . .''

\subsection{\LaTeX-Specific Advice}

Please use ``soft'' (e.g., \verb|\eqref{Eq}|) cross references instead
of ``hard'' references (e.g., \verb|(1)|). That will make it possible
to combine sections, add equations, or change the order of figures or
citations without having to go through the file line by line.

Please don't use the \verb|{eqnarray}| equation environment. Use
\verb|{align}| or \verb|{IEEEeqnarray}| instead. The \verb|{eqnarray}|
environment leaves unsightly spaces around relation symbols.

Please note that the \verb|{subequations}| environment in {\LaTeX}
will increment the main equation counter even when there are no
equation numbers displayed. If you forget that, you might write an
article in which the equation numbers skip from (17) to (20), causing
the copy editors to wonder if you've discovered a new method of
counting.

{\BibTeX} does not work by magic. It doesn't get the bibliographic
data from thin air but from .bib files. If you use {\BibTeX} to produce a
bibliography you must send the .bib files. 

{\LaTeX} can't read your mind. If you assign the same label to a
subsubsection and a table, you might find that Table I has been cross
referenced as Table IV-B3. 

{\LaTeX} does not have precognitive abilities. If you put a
\verb|\label| command before the command that updates the counter it's
supposed to be using, the label will pick up the last counter to be
cross referenced instead. In particular, a \verb|\label| command
should not go before the caption of a figure or a table.

Do not use \verb|\nonumber| inside the \verb|{array}| environment. It
will not stop equation numbers inside \verb|{array}| (there won't be
any anyway) and it might stop a wanted equation number in the
surrounding equation.

\subsection{Some Common Mistakes}\label{SCM}
\begin{itemize}
\item The word ``data'' is plural, not singular.
\item The subscript for the permeability of vacuum $\mu_{0}$, and other common scientific constants, is zero with subscript formatting, not a lowercase letter ``o''.
\item In American English, commas, semicolons, periods, question and exclamation marks are located within quotation marks only when a complete thought or name is cited, such as a title or full quotation. When quotation marks are used, instead of a bold or italic typeface, to highlight a word or phrase, punctuation should appear outside of the quotation marks. A parenthetical phrase or statement at the end of a sentence is punctuated outside of the closing parenthesis (like this). (A parenthetical sentence is punctuated within the parentheses.)
\item A graph within a graph is an ``inset'', not an ``insert''. The word alternatively is preferred to the word ``alternately'' (unless you really mean something that alternates).
\item Do not use the word ``essentially'' to mean ``approximately'' or ``effectively''.
\item In your paper title, if the words ``that uses'' can accurately replace the word ``using'', capitalize the ``u''; if not, keep using lower-cased.
\item Be aware of the different meanings of the homophones ``affect'' and ``effect'', ``complement'' and ``compliment'', ``discreet'' and ``discrete'', ``principal'' and ``principle''.
\item Do not confuse ``imply'' and ``infer''.
\item The prefix ``non'' is not a word; it should be joined to the word it modifies, usually without a hyphen.
\item There is no period after the ``et'' in the Latin abbreviation ``et al.''.
\item The abbreviation ``i.e.'' means ``that is'', and the abbreviation ``e.g.'' means ``for example''.
\end{itemize}
An excellent style manual for science writers is \cite{b7}.

\subsection{Authors and Affiliations}
\textbf{The class file is designed for, but not limited to, six authors.} A 
minimum of one author is required for all conference articles. Author names 
should be listed starting from left to right and then moving down to the 
next line. This is the author sequence that will be used in future citations 
and by indexing services. Names should not be listed in columns nor group by 
affiliation. Please keep your affiliations as succinct as possible (for 
example, do not differentiate among departments of the same organization).

\subsection{Identify the Headings}
Headings, or heads, are organizational devices that guide the reader through 
your paper. There are two types: component heads and text heads.

Component heads identify the different components of your paper and are not 
topically subordinate to each other. Examples include Acknowledgments and 
References and, for these, the correct style to use is ``Heading 5''. Use 
``figure caption'' for your Figure captions, and ``table head'' for your 
table title. Run-in heads, such as ``Abstract'', will require you to apply a 
style (in this case, italic) in addition to the style provided by the drop 
down menu to differentiate the head from the text.

Text heads organize the topics on a relational, hierarchical basis. For 
example, the paper title is the primary text head because all subsequent 
material relates and elaborates on this one topic. If there are two or more 
sub-topics, the next level head (uppercase Roman numerals) should be used 
and, conversely, if there are not at least two sub-topics, then no subheads 
should be introduced.

\subsection{Figures and Tables}
\paragraph{Positioning Figures and Tables} Place figures and tables at the top and 
bottom of columns. Avoid placing them in the middle of columns. Large 
figures and tables may span across both columns. Figure captions should be 
below the figures; table heads should appear above the tables. Insert 
figures and tables after they are cited in the text. Use the abbreviation 
``Fig.~\ref{fig}'', even at the beginning of a sentence.

\begin{table}[htbp]
\caption{Table Type Styles}
\begin{center}
\begin{tabular}{|c|c|c|c|}
\hline
\textbf{Table}&\multicolumn{3}{|c|}{\textbf{Table Column Head}} \\
\cline{2-4} 
\textbf{Head} & \textbf{\textit{Table column subhead}}& \textbf{\textit{Subhead}}& \textbf{\textit{Subhead}} \\
\hline
copy& More table copy$^{\mathrm{a}}$& &  \\
\hline
\multicolumn{4}{l}{$^{\mathrm{a}}$Sample of a Table footnote.}
\end{tabular}
\label{tab1}
\end{center}
\end{table}

\begin{figure}[htbp]
\centerline{\includegraphics{fig1.png}}
\caption{Example of a figure caption.}
\label{fig}
\end{figure}

Figure Labels: Use 8 point Times New Roman for Figure labels. Use words 
rather than symbols or abbreviations when writing Figure axis labels to 
avoid confusing the reader. As an example, write the quantity 
``Magnetization'', or ``Magnetization, M'', not just ``M''. If including 
units in the label, present them within parentheses. Do not label axes only 
with units. In the example, write ``Magnetization (A/m)'' or ``Magnetization 
\{A[m(1)]\}'', not just ``A/m''. Do not label axes with a ratio of 
quantities and units. For example, write ``Temperature (K)'', not 
``Temperature/K''.

\section*{Acknowledgment}

The preferred spelling of the word ``acknowledgment'' in America is without 
an ``e'' after the ``g''. Avoid the stilted expression ``one of us (R. B. 
G.) thanks $\ldots$''. Instead, try ``R. B. G. thanks$\ldots$''. Put sponsor 
acknowledgments in the unnumbered footnote on the first page.

\section*{References}

Please number citations consecutively within brackets \cite{b1}. The 
sentence punctuation follows the bracket \cite{b2}. Refer simply to the reference 
number, as in \cite{b3}---do not use ``Ref. \cite{b3}'' or ``reference \cite{b3}'' except at 
the beginning of a sentence: ``Reference \cite{b3} was the first $\ldots$''

Number footnotes separately in superscripts. Place the actual footnote at 
the bottom of the column in which it was cited. Do not put footnotes in the 
abstract or reference list. Use letters for table footnotes.

Unless there are six authors or more give all authors' names; do not use 
``et al.''. Papers that have not been published, even if they have been 
submitted for publication, should be cited as ``unpublished'' \cite{b4}. Papers 
that have been accepted for publication should be cited as ``in press'' \cite{b5}. 
Capitalize only the first word in a paper title, except for proper nouns and 
element symbols.

For papers published in translation journals, please give the English 
citation first, followed by the original foreign-language citation \cite{b6}.

\begin{thebibliography}{00}
\bibitem{b1} G. Eason, B. Noble, and I. N. Sneddon, ``On certain integrals of Lipschitz-Hankel type involving products of Bessel functions,'' Phil. Trans. Roy. Soc. London, vol. A247, pp. 529--551, April 1955.
\bibitem{b2} J. Clerk Maxwell, A Treatise on Electricity and Magnetism, 3rd ed., vol. 2. Oxford: Clarendon, 1892, pp.68--73.
\bibitem{b3} I. S. Jacobs and C. P. Bean, ``Fine particles, thin films and exchange anisotropy,'' in Magnetism, vol. III, G. T. Rado and H. Suhl, Eds. New York: Academic, 1963, pp. 271--350.
\bibitem{b4} K. Elissa, ``Title of paper if known,'' unpublished.
\bibitem{b5} R. Nicole, ``Title of paper with only first word capitalized,'' J. Name Stand. Abbrev., in press.
\bibitem{b6} Y. Yorozu, M. Hirano, K. Oka, and Y. Tagawa, ``Electron spectroscopy studies on magneto-optical media and plastic substrate interface,'' IEEE Transl. J. Magn. Japan, vol. 2, pp. 740--741, August 1987 [Digests 9th Annual Conf. Magnetics Japan, p. 301, 1982].
\bibitem{b7} M. Young, The Technical Writer's Handbook. Mill Valley, CA: University Science, 1989.
\end{thebibliography}


\end{document}
